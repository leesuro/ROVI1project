%Tracking
\input{data/tracking/qRobot1SlowTracking.txt}
\input{data/tracking/qRobot2SlowTracking.txt}
\input{data/tracking/qRobot3SlowTracking.txt}
\input{data/tracking/qRobot4SlowTracking.txt}
\input{data/tracking/qRobot5SlowTracking.txt}
\input{data/tracking/qRobot6SlowTracking.txt}
\input{data/tracking/qRobot7SlowTracking.txt}

\input{data/tracking/cameraPoseP0SlowTracking.txt}
\input{data/tracking/cameraPoseP1SlowTracking.txt}
\input{data/tracking/cameraPoseP2SlowTracking.txt}
\input{data/tracking/cameraPoseR0SlowTracking.txt}
\input{data/tracking/cameraPoseR1SlowTracking.txt}
\input{data/tracking/cameraPoseR2SlowTracking.txt}

\input{data/tracking/errorPoseXSlowTracking.txt}
\input{data/tracking/errorPoseYSlowTracking.txt}
\input{data/tracking/errorPoseXMediumTracking.txt}
\input{data/tracking/errorPoseYMediumTracking.txt}
\input{data/tracking/errorPoseXFastTracking.txt}
\input{data/tracking/errorPoseYFastTracking.txt}


\newcommand{\trackingSlowQRobotPlot}{
\label{plo:trackingQRobotPlot}
\begin{figure}[!ht]
\centering
	\tikzset{every mark/.append style={scale=0.5}}
	\begin{tikzpicture}
		\begin{axis}[height=9cm, width=\textwidth, grid=major,
		xlabel={Step},ylabel={rad}
		]	
			\addplot [color=Cyan, mark=o] coordinates {
				\trackingSlowQRobotDataA
			};
			\addlegendentry{q1}

			\addplot [color=DarkOrchid, mark=o] coordinates {
				\trackingSlowQRobotDataB
			};
			\addlegendentry{q2}
			
			\addplot [color=LimeGreen, mark=o] coordinates {
				\trackingSlowQRobotDataC
			};
			\addlegendentry{q3}
			
			\addplot [color=OrangeRed, mark=o] coordinates {
				\trackingSlowQRobotDataD
			};
			\addlegendentry{q4}
			
			\addplot [color=Goldenrod, mark=o] coordinates {
				\trackingSlowQRobotDataE
			};
			\addlegendentry{q5}
			
			\addplot [color=CarnationPink, mark=o] coordinates {
				\trackingSlowQRobotDataF
			};
			\addlegendentry{q6}

			\addplot [color=RoyalBlue, mark=o] coordinates {
				\trackingSlowQRobotDataG
			};
			\addlegendentry{q7}
		\end{axis}
	\end{tikzpicture}
	\caption{Robot's State while tracking the marker at slow speed.}
\end{figure}
}

\newcommand{\trackingSlowCameraPose}{
\label{plo:trackingSlowCameraPose}
\begin{figure}[!ht]
\centering
	\tikzset{every mark/.append style={scale=0.5}}
	\begin{tikzpicture} [spy using outlines=
	{circle, magnification=10, connect spies}
	]
		\begin{axis}[height=9cm, width=\textwidth, grid=major,
		xlabel={Step},ylabel={rad}
		]	
			\addplot [color=Cyan, mark=o] coordinates {
				\trackingSlowCameraPoseDataA
			};
			\addlegendentry{P0}

			\addplot [color=DarkOrchid, mark=o] coordinates {
				\trackingSlowCameraPoseDataB
			};
			\addlegendentry{P1}
			
			\addplot [color=LimeGreen, mark=o] coordinates {
				\trackingSlowCameraPoseDataC
			};
			\addlegendentry{P2}
			
			\addplot [color=OrangeRed, mark=o] coordinates {
				\trackingSlowCameraPoseDataD
			};
			\addlegendentry{R}
			
			\addplot [color=Goldenrod, mark=o] coordinates {
				\trackingSlowCameraPoseDataE
			};
			\addlegendentry{P}
			
			\addplot [color=CarnationPink, mark=o] coordinates {
				\trackingSlowCameraPoseDataF
			};
			\addlegendentry{Y}

			\coordinate (spypoint) at (axis cs:200,-0.31);
 			\coordinate (magnifyglass) at (axis cs:350,0.85);
		\end{axis}
		\spy [blue, size=2.0cm] on (spypoint) in node[fill=white] at (magnifyglass);
	\end{tikzpicture}
	\caption{Camera pose when tracking the marker at slow speed}
\end{figure}
}

\newcommand{\trackingErrorPlot}{
\label{plo:trackingErrorPlot}
\begin{figure}[!ht]
\centering
	\begin{tikzpicture}
		\begin{axis}[height=9cm, width=\textwidth, grid=major,
		xlabel={Step},ylabel={rad}
		]	
			\addplot [color=Cyan, mark=o] coordinates {
				\trackingSlowErrorPoseDataX
			};
			\addlegendentry{Slow Error X}

			\addplot [color=DarkOrchid, mark=o] coordinates {
				\trackingSlowErrorPoseDataY
			};
			\addlegendentry{Slow Error Y}
			
			\addplot [color=LimeGreen, mark=o] coordinates {
				\trackingMediumErrorPoseDataX
			};
			\addlegendentry{Medium Error X}
			
			\addplot [color=OrangeRed, mark=o] coordinates {
				\trackingMediumErrorPoseDataY
			};
			\addlegendentry{Medium Error Y}
			
			\addplot [color=Goldenrod, mark=o] coordinates {
				\trackingFastErrorPoseDataX
			};
			\addlegendentry{Fast Error X}
			
			\addplot [color=CarnationPink, mark=o] coordinates {
				\trackingFastErrorPoseDataY
			};
			\addlegendentry{Fast Error Y}
		\end{axis}
	\end{tikzpicture}
	\caption{Error in the X and Y coordinates. Tracking marker at different speeds}
\end{figure}
}


\chapter{Tracking points using the image Jacobian} % (fold)
\label{chap:tracking_points_using_the_image_jacobian}
The robot's inverse kinematics has been implemented as explained in the book and two features have been adapted respect to this. 
First, the input of the coordinates has been adapted for receive the OpenCV points and, second, the algorithm is able to track until three points (however, is easily expandable).

Despite it's possible to track three points, the algorithms developed for the feature extraction only return one point, so this three-point tracking has been only tested following the marker's frame. 
For this reason, only the experiments and conclusions with one-point tracking are going to be presented.

For the marker's frame tracking three sets of data are presented. 
First, the state of the robot (Q), second the camera's pose and third the error obtained between the real frame movement and the calculated from the camera.

It is a requirement show when the robot lost the marker but in all this cases the robot was able to track the marker and no over-speed was reached. 
For the Q and the camera's pose only the data from the slow-speed marker is going to be shown. 
This is due to the results are the same for all the speeds but in the slow there is more information.
In the figure \ref{fig:trackingSlowQRobotPlot} the robot's state is presented while the same is done in the figure \ref{fig:trackingSlowCameraPosePlot} with camera's pose.

	\ifx \plots \yes
		\trackingSlowQRobotPlot 
	\fi
	\ifx \plots \yes
		\trackingSlowCameraPosePlot 
	\fi

As can be seen, the great quantity of experiments make the different values look like a line, but in \ref{fig:trackingSlowCameraPosePlot} an magnifying glass show that there are actually different points.

Both plots suggest smooth movements of the robot what can explain that the maximum velocities are not reached

Lastly, the error is calculated as the difference between the real movement gotten from the data sets given and the movement calculated for a point seen from the virtual camera.
This error occurs in the X and Y coordinates and it differs depending on the velocity's marker so all the errors are different and the results are presented in the figure \ref{fig:trackingErrorPlot}.

	\ifx \plots \yes
		\trackingErrorPlot
	\fi
	
In the figure \ref{fig:trackingErrorPlot} can be distinguished different areas depending on the speed that are differentiated with a jump between them. 
This areas correspond with a change in the direction of the robot and they show how the robot "vibrate" until the desired position is reached.

The figure shows how the error has a natural damping that finally converge in a stable value. This is a second order dynamic system that has a characteristic stabilization time and dumping ratio. What can be deduced from the data is that, for a faster speed the errors are bigger as the stabilization time. While in the fast movements the error reach almost the 60 pixels of error, in the slow movement it is under ten. Also, related with the stabilization time, only the slow movements reach an stable value before change the direction.

This dumping has been controlled from the code multiplying the dQ by a percentage. The experiments results show that the dumping is heavily controlled if a 95\% is applied but this makes the robots move less than expected. We have found a good balance between achieve objectives and stabilize the robot in a 97\% of dQ applied.

\qRobotThreeTrackingMediumPlot
\cameraPoseThreeTrackingMediumPlot
\errorThreeTrackingPlot 

% chapter tracking_points_using_the_image_jacobian (end)